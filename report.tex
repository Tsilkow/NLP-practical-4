% --------------------------------------------------------------
% This is all preamble stuff that you don't have to worry about.
% Head down to where it says "Start here"
% --------------------------------------------------------------
 
\documentclass[12pt]{article}

\usepackage[margin=1in]{geometry} 
\usepackage{amsmath,amsthm,amssymb,mathtools}
\newcommand{\shellcmd}[1]{\\\indent\indent\texttt{\footnotesize\# #1}\\}
\usepackage[T1]{fontenc}
\usepackage[polish]{babel}
\usepackage[utf8]{inputenc}

\def\[#1\]{\begin{align*}#1\end{align*}} 

\begin{document}
 
% --------------------------------------------------------------
%                         Start here
% --------------------------------------------------------------
 
\title{Practical No. 4 }%replace X with the appropriate number
\author{Tomasz Siłkowski\\ %replace with your name
ts407106@students.mimuw.edu.pl}
 
\maketitle

\setcounter{section}{1}
\section{Attention Exploration}
\subsection*{a) Attention input for identity}
For attention to approximately copy one of the value vectors \(v_i\), \(k^T_i q\)
would have to significantly greater than \(k^T_j q\) for any \(j \neq i\).

\medskip

\subsection*{b) Query input for attention to return average }
\[
q = C (k_a + k_b), \text{where } C \gg 0
\]

\medskip

\subsection*{c) Drawbacks of single-headed attention }
\begin{enumerate}
\item \[
q = C (\mu_a + \mu_b), \text{where } C \gg 0
\]
Because covariance matrices for all keys \(k\) are identity matrix multiplied by a vanishingly small
constant, we know that keys (random variables) \(k_a\) and \(k_b\) are not correlated, therefore
independent. That means \(\mu_a\) and \(\mu_b\) are their best estimators and are also perpendicular
to each other, they are the best guess for requested value. \\
\item
To restate the problem, we have \(k_a \sim \mathcal{N}(\mu_a, \alpha I + \frac{1}{2}(\mu_a\mu^T_a))\)
where \(\alpha\) is vanishingly small. \\
Let \(q = C (\mu_a + \mu_b)\) (same as part 1) and let \(k_b\) be the key pointing in the same
direction as \(k_a\) but with different norm. This means:
\[
k^T_i q \approx
\begin{cases*}
  \varepsilon_a C & for \(i = a\), where \(\varepsilon_a \sim \mathcal{N}(1, \frac{1}{2})\) \\
  \varepsilon_b C & for \(i = b\), where \(\varepsilon_b \sim \mathcal{N}(1, \frac{1}{2})\) \\
  0 & otherwise
\end{cases*}
\]
Because of that, when calculating \(c\):
\[
c &= \frac{\exp(\varepsilon_a C)}{\exp(\varepsilon_a C) + \exp(\varepsilon_b C)} v_a +
\frac{\exp(\varepsilon_b C)}{\exp(\varepsilon_a C) + \exp(\varepsilon_b C)} v_b = \\
&= \frac{1}{\exp((\varepsilon_a - \varepsilon_b) C)} v_a + \frac{1}{\exp((\varepsilon_b - \varepsilon_a) C)} v_b.
\]
Because \(\varepsilon_a\) and \(\varepsilon_b\) come from the same distribution, it is
equally likely that \(c\) will be closer to \(v_a\) as to \(v_b\). This means \(c\) will be closer
to
whichever \(v_i\) has bigger \(|k_i|\) for \(i \in \{a, b\}\).
\end{enumerate}

\medskip

\subsection*{d) Benefits of multi-headed attention }
\begin{enumerate}
\item
\[
q_1 &= C_1 \mu_a, \text{where } C_1 \gg 0 \\
q_2 &= C_2 \mu_b, \text{where } C_2 \gg 0
\]
\item
\[
k^T_a q_1 &= C_1 \varepsilon_a \\
k^T_b q_2 &= C_2 \varepsilon_b \\
\intertext{Then: }
c_1 &= v_a \\
c_2 &= v_b \\
c = \frac{1}{2}(c_1 + c_2) &\approx \frac{1}{2}(v_a + v_b)
\]
Basing my judgement on these calculations, I expect output \(c\) to be close to the average of
\(v_a\) and \(v_b\).
\end{enumerate}

\subsection*{e) Key-Query-Value self-attention intuition }
\begin{enumerate}
  \item
    \(c_2\) approximates vector \(u_a\). \\
    It's impossible to approximate \(u_b\) with \(c_2\) by adding either \(u_c\) or \(u_d\) to \(x_2\).
    Any of these vectors will increase in value equally along with \(u_b\). That means there's no way
    for \(u_b\) to dominate the combination. \\
  \item
    
\section{Pretraining Transformer-based Generative Model}
\subsection*{d) Model with only finetuning}
After 75 epochs of finetuning previously untrained model, in evaluation it achieved a score of 1.2\% (6 out of 500 correct). \\
For comparison, naive model (outputting only ``London'' as an answer) achieved a score of 5.0\% (25 out of 500 correct). \\

%  \smallbreak
% --------------------------------------------------------------
%     You don't have to mess with anything below this line.
% --------------------------------------------------------------
 
\end{document}
